\part{Tools}
\label{part:tools}

\chapter{Tool Overview}

\section{Symbolic execution}

The tools currently in the CIVL tool kit all use \emph{symbolic
  execution}.  This is a technique in which variables are assigned
symbolic rather than concrete values.  In particular, input variables
are assigned unique \emph{symbolic constants}, which are symbols such
as $X$, $Y$, and so on.  Operations produce symbolic expressions in
those symbols, such as $(X+Y)/2$.

\section{Commands}

Current tools allow one to \emph{run} a CIVL program using random
choice to resolve nondeterministic choices; \emph{verify} a program
using model checking to explore all states of the program; and
\emph{replay} a trace if an error is found.  There are also commands
to show the results just of preprocessing or parsing a file; as these
are basically sub-tasks of the other tasks, they are used mainly for
debugging.

Each tool is launched from the command line by typing ``\texttt{civl}
\textit{cmd} \ldots'', where \textit{cmd} is one of
\begin{itemize}
\item \ct{help} : print usage information
\item \ct{run}  : run the program using random simulation
\item \ct{verify} : verify program
\item \ct{replay} : replay trace for program
\item \ct{parse} : show result of preprocessing and parsing file
\item \ct{preprocess} : show result of preprocessing file only.
\end{itemize}
The additional arguments and options are described below and are also
shown by the \ct{help} command.

A number of properties are checked when running or verifying
a CIVL program.  These include the following:
\begin{itemize}
\item absence of deadlocks
\item absence of assertion violations
\item absence of division or modulus with denominator $0$
\item absence of illegal pointer dereferences
\item absence of out-of-bounds array indexes
\item absence of invalid casts
\item every object is defined (i.e., initialized) before
  it is used.
\end{itemize}


\section{Options}

The following command line options are available:

\newenvironment{optionlist}{\begin{flushleft}}{\end{flushleft}}
\newcommand{\option}[2]{#1\\[0mm]\makebox[5mm]{}#2\\[3mm]}
\newcommand{\booloption}[3]{\option{\ct{-#1} or \ct{-#1=}\textit{boolean}
    (default: \ct{#2})}{#3}}
\newcommand{\intoption}[3]{\option{\ct{-#1=}\textit{integer}
    (default: \ct{#2})}{#3}}
\newcommand{\mapoption}[2]{\option{\ct{-#1}\textit{key}\ct{=}\textit{value}}{#2}}
\newcommand{\stringoption}[3]{\option{\ct{-#1=}\textit{string}
    (default: \ct{#2})}{#3}}

\begin{optionlist}
  \booloption{debug}{false}{debug mode: print very detailed information}
  \booloption{echo}{false}{print the command line}
  \booloption{enablePrintf}{true}{enable \texttt{printf} function}
  \intoption{errorBound}{1}{stop after finding this many errors}
  \booloption{guided}{false}{user guided simulation; applies only to
    run, ignored for all other commands}
  \intoption{id}{0}{ID number of trace to replay}
  \mapoption{input}{initialize input variable \textit{key} to
    \textit{value}}
  \intoption{maxdepth}{2147483647}{bound on search depth}
  \booloption{min}{false}{search for minimal counterexample}
  \stringoption{por}{std}{partial order reduction (por) choices:
    std (standard por) or scp (scoped por)}
  \booloption{random}{varies}{select enabled transitions randomly;
    default for \texttt{run}, ignored for all other commands}
  \booloption{saveStates}{true}{save states during depth-first search}
  \stringoption{seed}{none}{set the random seed; applies only to run}
  \booloption{showAmpleSet}{false}{print the ample set of each state}
  \booloption{showModel}{false}{print the model}
  \booloption{showProverQueries}{false}{print theorem prover queries
    only}
  \booloption{showQueries}{false}{print all queries}
  \booloption{showSavedStates}{false}{print saved states only}
  \booloption{showStates}{false}{print all states}
  \booloption{showTransitions}{false}{print transitions}
  \booloption{simplify}{true}{simplify states?}
  \booloption{solve}{false}{try to solve for concrete counterexample}
  \stringoption{sysIncludePath}{}{set the system include path}
  \stringoption{trace}{}{filename of trace to replay}
  \stringoption{userIncludePath}{}{set the user include path}
  \booloption{verbose}{false}{verbose mode}
\end{optionlist}

\section{Errors}
\label{sec:errors}

When a property violation occurs, either in \emph{verification} or
\emph{run} mode, a brief report is written to the screen.  In
addition, a report may be \emph{logged} in the directory
\texttt{CIVLREP}.

The \emph{error bound} parameter determines how many errors can be
encountered before a search terminates.  By default, the error bound
is 1, meaning a search will stop as soon as the first error is found.
The error bound can be set to a higher number on the command line
using option \emph{errorBound}.

When the error bound is greater than one, the CIVL verifier continues
searching after the first error is discovered.  It first attempts to
``recover'' from the error by adding to the path condition a clause
which guarantees that the error cannot happen.  For example, if the
error was caused by a possible division by zero, $x/y$, where $y$ is
an unconstrained real symbolic constant, CIVL will add to the path
condition the predicate $y\neq 0$, and continue the search.  In some
cases, CIVL determines that the modified path condition is
unsatisfiable, in which case the search will back-track in the usual
symbolic execution way.

In addition to the printed reports, errors are logged.  However, CIVL
follows a protocol aimed at limiting the number of reports of errors
which are essentially the same.  This protocol uses a simple
equivalence relation on the set of errors.  Two erroneous states are
considered equivalent if the errors are of the same \emph{kind}
(deadlock, division by zero, illegal pointer dereference, etc.) and
every process is at the same location in both states.  When an error
is encountered, CIVL first checks to see if an earlier equivalent
errors exists in the log.  If so, the lengths of the traces leading to
the two error states are compared.  If the new trace is shorter, the
old log entry is replaced with the new one.  In this way, only the
shortest representative error trace for each equivalence class of
errors is recorded in the log.

A log entry actually entails two things: first, a plain text entry
similar to the one printed to the screen is made in a log file in
\texttt{CIVLREP}.  The name of this file is usually of the form
\texttt{\textit{root}{\U}log.txt}, where \textit{root} is the root of
the original file name, i.e., the substring of the file name that ends
just before the first `\texttt{.}'.  For example, if the file name is
\texttt{diningBad.cvl}, the log file will be named
\texttt{diningBad{\U}log.txt}.  This is a plain text, human-readable
file which summarizes the results of the verification run.

In addition, each saved trace is stored in a separate file in
\texttt{CIVLREP}.  The names of these files have the form
\texttt{\textit{root}{\U}\textit{id}.trace}, where \textit{id} is the
error ID (reported when the error is printed and logged).  This file
is not intended to be human-readable.  It contains a compressed
representation of the trace, including all of the options and
parameter values and choices made a nondeterministic points.  
This file is used by CIVL when replaying an error trace.

As mentioned above, the CIVL \texttt{replay} command is used to play
an earlier-saved error trace.  When more than one trace has been
saved, the \texttt{-id} command line option can be used to specify
which one to play. (The default \emph{id} is 0).

\chapter{Interpreting the Output}

\section{Transitions}

Transitions are printed during trace \texttt{replay}, in the course of
a \texttt{run}, or during verification if option
\emph{showTransitions} is selected.  A typical transition is printed
as follows:

\begin{verbatim}
State 6, proc 0: 
  44->49: LOOP_FALSE_BRANCH at f0:36.18-23 "i < n";
  49->RET: return (init) at f0:37.0-1 "}";
  7->8: i = 0 at f0:42.7-16 "int i = 0";
--> State 74
\end{verbatim}

This means that the transition begins executing from the state with
ID 6, and it is executed by the process with PID 0.  The transition
is executed in a sequence of atomic steps; in this case there the
transition consists of three steps.

Process 0 begins at location 44; this is a static location in the
program graph of a function in the CIVL model.  Details about the
locations can be seen by printing the CIVL model.  In executing the
first step, control moves from location 44 to location 49.  This first
step is an edge in the program graph corresonding to the \emph{false}
branch of a loop condition, i.e., the branch that exits the loop.

The remainder of the line describing the step specifies the part of
the original source code corresponding to this step.  The source code
fragment can be found in file \texttt{f0}.  To save space and avoid
constantly repeating long paths, all the source files involved in a
program are printed once and assigned keys such as \texttt{f0},
\texttt{f1}, etc.  The legend is printed out once at the beginning of
the run; in this case it is simply the following:
\begin{verbatim}
File name list:
f0	: dining.cvl
\end{verbatim}
The source code fragment begins on character 18 of line 36 of
\texttt{f0}, and extends to character 23 of that line.  This range is
inclusive on the left and exclusive on the right, so the total number
of characters in this range is $23-18=5$.  The five characters from
the source code are printed next inside double quotes.  For longer
ranges, this excerpt will be abbreviated using an elipsis.

The second step executes a \emph{return} statement, which results in
popping the top activation frame from process 0's call stack. The
function returning is \texttt{init}.  Since the program counter for
that frame disappears with the execution of this step, there is no
final value for its new location; this is signified using the
pseudo-location \texttt{RET}.

In the new top frame, control is at location 7, and an assignment
statement is executed, moving control to location 8.  With this last
step, the transition ends at State 74.

Between transitions, processes can be renumbered.  Hence a process
with PID 0 in one state, may have a different PID in another state.
The same is true for dynamic scope IDs.  Within a single transition,
however, these numbers will not change.

\section{States}

States are printed typically when a property is violated, at the
initial or final points of a trace replay, or if the option
\emph{showStates}, \emph{showSavedStates}, \emph{verbose}, or
\emph{debug} is selected.

\begin{figure}
  \begin{small}
\begin{verbatim}
State 7
| Path condition
| | 0 <= SIZEOF(dynamicType<146>)+-1
| Dynamic scopes
| | dyscope d0 (id=0, parent=d0, static=0)
| | | reachers = {0}
| | | variables
| | | | __atomic_lock_var = process<-1>
| | | | __heap = NULL
| | dyscope d1 (id=1, parent=d0, static=4)
| | | reachers = {0}
| | | variables
| | | | __heap = $heap<(A[1][])<H_p0s1v0i0l0[0:=A<(H_p0s1v0i0l0[0].0)[0:=1, 1:=2, 2:=3],2>]>>
| | | | a = &heapObject<d1,0,0>[0]
| | | | b = &heapObject<d1,0,0>[0].a[2]
| Process states
| | process p0 (id=0)
| | | atomicCount=0
| | | call stack
| | | | Frame[function=_CIVL_system, location=14, f0:25.2-9 "$assert", dyscope=d1]
\end{verbatim}
  \end{small}
  \caption{Complete print-out of a state}
  \label{fig:state-print}
\end{figure}

A complete print-out of a state can be seen in Figure
\ref{fig:state-print}.  The state is presented in hierarchical way.
At the top-most level of this hierarchy, there are 3 main parts to the
state:
\begin{enumerate}
\item the \emph{path condition}, i.e., the boolean-valued symbolic
  expression used in symbolic execution to keep track of all
  conditions on the input symbols which must hold in order for the
  current path to have been followed;
\item the \emph{dynamic scopes}, and
\item the \emph{process states}.
\end{enumerate}

The dynamic scopes are numbered starting from 0.  This numbering is
arbitrary and is invisible to the program, i.e., there is no way for
the program to obtain its dynamic scope ID.  This allows the verifier
to renumber dynamic scopes at will in order to transform a state into
an equivalence canonical form.

The print-out of each dynamic scope specifies the ID of the dyscope's
parent in the dyscope tree.  (The root dyscope shows $-1$ for the
parent.)  This specifies the complete tree structure of the dyscopes.
Each dyscope is an instance of some static scope; the representation
also shows the ID of this static scope.

The next line in the representation of the dyscope shows a set of
\emph{reachers}.  These are the PIDs of the processes that can
\emph{reach} this dyscope.  A process can reach a dyscope if there is
path in the dyscope tree that starts from a dyscope referenced by a
frame on the process' call stack and follows the parent edges in the
tree.

The \emph{variables} section of the dyscope representation consists of
one line for each variable in the static scope corresponding to the
dyscope.  There are also special hidden variables, such as the heap.
In each case, the value assigned to the variable is shown.  A value of
\texttt{NULL} indicates that the variable is currently undefined.  The format for the
value of a pointer depends on the type of object being referenced, as 
follows:
\begin{itemize}
\item A variable: {\tt \&}\emph{variable} {\tt <}\emph{dyscope name}{\tt >}
\item An element of an array: {\tt \&}\emph{array} {\tt <}\emph{dyscope name}{\tt >[}\emph{index}{\tt]}
\item A struct field: {\tt \&}\emph{variable} {\tt <}\emph{dyscope name}{\tt >.}\emph{field}
\item A heap cell: \emph{variable} {\tt <}\emph{dyscope name}{\tt,} \emph{malloc ID}{\tt,} \emph{malloc call number}{\tt >}
\end{itemize}

The process states section consists of one sub-section for each
process currently in the state.  Like the dynamic scopes, the
processes are numbered in some arbitrary way. For each process, the
value of the \emph{atomic count} is given.  This is the nesting depth
of the atomic blocks in which the process is currently located, i.e.,
the number of times the process has entered an atomic block without
exiting the block.

The call stack of the process lists the activation frames on the stack
from top to bottom.  The frame at the top correspond to the function
currently executing in that process.  The name of the function, the
value of the program counter (location), and the source code for that
location, and the dyscope ID for the frame are shown.

\section{Property Violations}

As described in Section \ref{sec:errors}, an error report is printed
whenever CIVL encounters an error. A typical error report appears as
follows:
\begin{small}
\begin{verbatim}
Error 0 encountered at depth 21:
CIVL execution error (kind: DEADLOCK, certainty: PROVEABLE)
A deadlock is possible:
  Path condition: true
  Enabling predicate: false
ProcessState 0: terminated
ProcessState 1: at location 25, f0:21.30-42 "forks[right]"
  Enabling predicate: false
ProcessState 2: at location 25, f0:21.30-42 "forks[right]"
  Enabling predicate: false
at f0:21.30-42 "forks[right]".
\end{verbatim}
\end{small}

The report begins with ``Error 0''.  The errors are numbered in the
order they are discovered in this search; this indicates that this is
the first (0th) error encountered.  The depth refers to the length of
the depth-first search stack when the error was encountered.  In this
case, the depth was 21, meaning that the trace leading to the
erroneous state is a sequence of 21 states and 20 transitions.

The errors are categorized by \emph{kind}.  The error kinds include
\emph{deadlock}, indicating that it is possible no transition is
enabled in the state; \emph{assertion violation}, indicating an
assertion may fail in the state; \emph{division by zero}; and
\emph{out of bounds}, indicating an array index may be out of bounds,
among many more.

In addition to the brief report shown above, most error reports
also include a complete print-out of the state at which the error
occurred.  They will also include additional information specific
to the kind of error.  For example, the deadlock error report shown
above includes the following information:
\begin{itemize}
\item the value of the path condition;
\item the \emph{enabling predicate}, which is the disjunction of the
  guards associated to all transitions departing from the current
  state. This is the predicate that CIVL has found to possibly be
  unsatisfiable under the context of the path condition; and
\item for each process, the current location of the process and the
  enabling predicate for that process, i.e., the disjunction of the
  guards associated to all transitions departing from the current
  state in that process (CIVL has found that all of these may be
  unsatisfiable).
\end{itemize}

Errors are also categorized as to their \emph{certainty}.  CIVL is
\emph{conservative}, meaning that if it not sure a property holds in a
state, it will report it.  This means that it may sometimes raise
\emph{false alarms}, i.e., report a possible error even when none
exists. The certainty measures how certain CIVL is that the error is
real.  The certainty levels, from most to least certain, are as
follows:
\begin{enumerate}
\item \emph{concrete}: this indicates that CIVL has actually found
  concrete
  values for all input variables that are guaranteed to drive the
  execution along the current trace and result in the error;
\item \emph{proveable}: this indicates that a theorem prover (either
  the external one or CIVL's built-in prover) has determined that
  the error is feasible, which includes proving that the path condition
  is satisfiable; however, it has not necessarily found concrete
  values for the inputs;
\item \emph{maybe}: this indicates the prover is not sure whether this
  is an error; this could be due to the incompleteness of the decision
  procecure, or it could be a real error;
\item \emph{none}: probably an internal CIVL error: the theorem prover
  has not said anything.
\end{enumerate}

\section{Statistics}

\begin{itemize}
\item \emph{validCalls}: the number of calls to the CIVL \emph{valid}
  method, used to determine if a first-order formula is valid under
  a given first-order context.   Some of these queries are resolved
  quickly by CIVL; when CIVL cannot resolve the query itself, it
  calls a separate theorem prover (CVC3)
\item \emph{proverCalls}: the number of calls to the separate theorem
  prover's \emph{valid} method
\item \emph{memory}: the total amount of memory, in bytes, consumed by
  the Java virtual machine at the end of the search/run.
\item \emph{time}: the total time, in seconds, used to perform the
  CIVL operation
\item \emph{maxProcs}: the maximum process count achieved, over all states 
  encountered in the search/run
\item \emph{statesInstantiated}: the number of state objects
  instantiated during the course of the verification/run
\item \emph{statesSaved}: the number of states saved in the course
  of a search
\item \emph{statesSeen}: the number of states pushed onto the 
  depth-first search stack in the course of the search; note
  that ``intermediate states'' created in the process of executing
  a transition are not pushed onto the stack, only the final
  state resulting from the transition is pushed
\item \emph{statesMatched}: the number of times a state encountered
  in the depth-first search was found to match a saved state seen
  earlier in the search
\item \emph{steps}: the total number of primitive steps executed
  during the verification/run.  A step is the smallest, atomic,
  unit of execution; each transition is composed of one or more steps.
  This number is a good measure for the total amount of ``work''
  carried out by CIVL
\item \emph{transitions}: the total number of transitions executed
  during the verification/run.  A transition is a coarser unit
  of execution; each transition consists of one or more steps 
  executed from a single process, resulting in a state which is then
  pushed onto the DFS stack.
\end{itemize}

\chapter{Emacs mode}

A CIVL-C mode for the \texttt{emacs} text editor is available in
directory \texttt{emacs} in the CIVL distribution.  This provides
syntax highlighting and automatic indentation for CIVL-C programs.

To install this mode:
\begin{enumerate}
\item Copy file \texttt{civl-syntax.el} to \verb!~/.emacs.d/lisp! or another favorite location
\item Include that path in your load path in \verb!.emacs!:
\begin{verbatim}
(add-to-list 'load-path "~/.emacs.d/lisp")
\end{verbatim}
\item Add the following lines to your \verb!~/.emacs! file:
\begin{verbatim}
(require 'civl-syntax)
(civl-syntax)
\end{verbatim}
\end{enumerate}

We are grateful to William Killian of the University of Delaware for
writing this emacs module.
