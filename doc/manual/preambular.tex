% number subsubsections and put them in the table of contents...
\setcounter{tocdepth}{3}
\setcounter{secnumdepth}{3}

\newcommand{\notimp}{\emph{Status:} Not yet implemented.}

% algorithm2e package settings...

\SetInd{1mm}{2mm}
\SetVlineSkip{.5mm}
\SetKw{Break}{break}
\SetKw{Assert}{assert}
\SetKw{Integer}{Integer}
\SetKw{Array}{array}
\SetKw{Of}{of}
\SetKwFor{Procedure}{procedure}{is}{end procedure}
\SetKwInOut{Uses}{uses}
\SetKwInOut{Invariant}{invariant}

% Type/font styles...
\newcommand{\upsty}[1]{\textup{\textsf{#1}}}
\newcommand{\setsty}[1]{\textup{\textsf{\textcolor{blue}{#1}}}}
\newcommand{\code}[1]{\texttt{\textcolor{blue}{#1}}}
\newcommand{\ct}[1]{\texttt{#1}}

% Theorem environments...
\newtheorem{theorem}{Theorem}
\theoremstyle{definition}
\newtheorem{definition}{Definition}[subsection]

% Symbols...
\renewcommand{\implies}{\Rightarrow}
\newcommand{\ra}{\rightarrow}
\newcommand{\B}{\mathbb{B}}
\newcommand{\N}{\mathbb{N}}
\newcommand{\lb}{\texttt{\char`\{}}
\newcommand{\rb}{\texttt{\char`\}}}
\newcommand{\U}{\texttt{\char`\_}}
%\newcommand{\U}{{\ttfamily\symbol{'137}}}
\newcommand{\UU}{\U\U}

% Static notions...
\newcommand{\rootscope}{\setsty{root}} % root static scope
\newcommand{\start}{\setsty{start}} % start location $l_0$
\newcommand{\Var}{\setsty{Var}} % set of all variables
\newcommand{\bool}{\setsty{bool}} % boolean type
\newcommand{\proc}{\setsty{proc}} % process reference type
\newcommand{\Val}{\setsty{Val}} % set of all values
\newcommand{\vtype}{\setsty{vtype}} % type of variable
\newcommand{\etype}{\setsty{etype}} % type of expression
\newcommand{\Type}{\setsty{Type}} % set of all types
\newcommand{\Expr}{\setsty{Expr}} % set of all expressions
\newcommand{\Eval}{\setsty{Eval}} % set of all valuations on variables
\newcommand{\eval}{\setsty{eval}} % evaluation of an expression
\newcommand{\sparent}{\setsty{sparent}} % parent of a static scope
\newcommand{\dparent}{\setsty{dparent}} % parent of a dynamic scope
\newcommand{\fscope}{\setsty{fscope}} % scope of a function
\newcommand{\lscope}{\setsty{lscope}} % scope of a location
\newcommand{\static}{\setsty{static}} % static scope associated to a dynamic one
\newcommand{\stack}{\setsty{stack}} % call stack
\newcommand{\vars}{\setsty{vars}} % variables declared in a scope
\newcommand{\ancestors}{\setsty{ancestors}} % as in tree
\newcommand{\descendants}{\setsty{descendants}} % as in tree
\newcommand{\returntype}{\setsty{returnType}} % function's return type
\newcommand{\numparams}{\setsty{numParams}} % number of params for a func.
\newcommand{\params}{\setsty{params}} % sequence of formal parameters
\newcommand{\void}{\setsty{void}} % type for func. returning nothing
\newcommand{\Loc}{\setsty{Loc}} % locations in function's trans. sys.
\newcommand{\Func}{\mathcal{F}} % set of function symbols
\newcommand{\true}{\textit{true}} % boolean value true
\newcommand{\false}{\textit{false}} % boolean value false
\newcommand{\default}{\setsty{default}} % default value of type
\newcommand{\len}{\setsty{length}} % length of sequence
\newcommand{\func}{\setsty{func}} % function a static scope belongs to

% Dynamic notions...
\newcommand{\fnode}{\setsty{fnode}} % function node of dyscope
\newcommand{\State}{\setsty{State}} % set of all states of model
\newcommand{\deval}{\setsty{deval}} % valuation on dyscope
\newcommand{\droot}{\setsty{droot}} % root dyscope
\newcommand{\Frame}{\setsty{Frame}} % set of activation frames

% Tables...
\newcommand{\notationtable}{%
  \begin{tabular}{lll}
    Symbol & Section & Meaning\\ \hline
    $\B$ & \S\ref{sec:notation} & \{\true,\false\}\\
    $\N$ & \S\ref{sec:notation} & \{0,1,2,\ldots\}\\
    $\ancestors$ & \S\ref{sec:notation} & set of ancestors of node in a tree (inclusive)\\
    $\descendants$ & \S\ref{sec:notation} & set of descendants of node
    in a tree (inclusive)\\
    $\len$ & \S\ref{sec:notation} & length of a sequence\\
    $\Var$ & \S\ref{sec:context} & the set of all variables\\
    $\bool$ & \S\ref{sec:context} & the boolean type\\
    $\proc$ & \S\ref{sec:context} & the process reference type\\
    $\Val$ & \S\ref{sec:context} & the set of all values\\
    $\Val_t$ & \S\ref{sec:context} & values of type $t$\\
    $\default_t$ & \S\ref{sec:context} & default value of type $t$\\
    $\vtype$ & \S\ref{sec:context} & function $\Var\ra\Type$ giving type of each variable\\
    $\Eval$ & \S\ref{sec:context} & set of all valuations on $\Var$\\
    $\Eval(V)$ & \S\ref{sec:state} & set of all valuations on variables in $V\subseteq\Var$\\
    $\Expr$ & \S\ref{sec:context} & set of typed expressions over $\Var$\\
    $\etype$ & \S\ref{sec:context} & $\Expr\ra\Type$, gives type of each expression\\
    $\eval$ & \S\ref{sec:context} & $\Expr\times\Eval\ra\Val$, the evaluation function\\
    $\mathcal{C}$ & \S\ref{sec:context} & a CIVL context\\
    $\Sigma$ & \S\ref{sec:scopes} & set of all static scopes\\
    $\rootscope$ & \S\ref{sec:scopes} & the root scope (member of $\Sigma$)\\
    $\sparent$ & \S\ref{sec:scopes} & $\Sigma\setminus\{\rootscope\}\ra\Sigma$, parent function in static scope tree\\
    $\vars$ & \S\ref{sec:scopes} & $\Sigma\ra 2^{\Var}$, specifies variables declared in scope\\
    $\Lambda$ & \S\ref{sec:scopes} & a lexical scope system\\
    $\void$ & \S\ref{sec:functions} & type used for function that does not return  a value\\
    $\Type'$ & \S\ref{sec:functions} & $\Type\cup\{\void\}$\\
    $\Func$ & \S\ref{sec:functions} & set of function symbols\\
    $\fscope$ & \S\ref{sec:functions} & $\Func\ra\Sigma\setminus\{\rootscope\}$, gives function scope of each function\\
    $\returntype$ & \S\ref{sec:functions} & $\Func\ra\Type'$, gives return type of each function\\
%    $\numparams$ & \S\ref{sec:functions} & $\Func\ra\N$, gives number of inputs for each function\\
    $\params$ & \S\ref{sec:functions} & $\Func\ra\Var^*$, formal
    parameter sequence for $f\in\Func$\\
    $f_0$ & \S\ref{sec:functions} & the root function (member of
    $\mathcal{F}$)\\
    $\func$ & \S\ref{sec:functions} & $\Sigma\ra\Func$, function to
    which scope belongs\\
    $\Loc_f$ & \S\ref{sec:gts} & set of locations for $f\in\Func$\\
    $\lscope_f$ & \S\ref{sec:gts} & $\Loc_f\ra\Sigma$, gives scope of each location for $f\in\Func$\\
    $\start_f$ & \S\ref{sec:gts} & start location for $f\in\Func$ (member of $\Loc_f$)\\
    $T_f$ & \S\ref{sec:gts} & set of guarded transitions for $f\in\Func$\\
  \end{tabular}
}


% CIVL-C keywords

\newcommand{\cckey}{\$}
\newcommand{\cc}[1]{\mbox{\texttt{\cckey{}#1}}}
\newcommand{\cabstract}{\cc{abstract}}
\newcommand{\cproc}{\cc{proc}}
\newcommand{\cprocdefined}{\cc{proc{\U}defined}}
\newcommand{\cprocNull}{\cc{proc{\U}null}}
\newcommand{\cself}{\cc{self}}
\newcommand{\cinput}{\cc{input}}
\newcommand{\coutput}{\cc{output}}
\newcommand{\cspawn}{\cc{spawn}}
\newcommand{\cwait}{\cc{wait}}
\newcommand{\cwaitall}{\cc{waitall}}
\newcommand{\cassert}{\cc{assert}}
\newcommand{\ctrue}{\cc{true}}
\newcommand{\cfalse}{\cc{false}}
\newcommand{\cassume}{\cc{assume}}
\newcommand{\catom}{\cc{atom}}
\newcommand{\catomic}{\cc{atomic}}
\newcommand{\cwhen}{\cc{when}}
\newcommand{\cchoose}{\cc{choose}}
\newcommand{\cchooseint}{\cc{choose{\U}int}}
\newcommand{\cinvariant}{\cc{invariant}}
\newcommand{\crequires}{\cc{requires}}
\newcommand{\censures}{\cc{ensures}}
\newcommand{\cexit}{\cc{exit}}
\newcommand{\cresult}{\cc{result}}
\newcommand{\cat}{\texttt{@}}
\newcommand{\ccollective}{\cc{collective}}
\newcommand{\cfor}{\cc{for}}
\newcommand{\cparfor}{\cc{parfor}}

%\newcommand{\cheap}{\cc{heap}}
\newcommand{\cscope}{\cc{scope}}
\newcommand{\cscopedefined}{\cc{scope{\U}defined}}
\newcommand{\cscopeof}{\cc{scopeof}}
\newcommand{\cscopeparent}{\cc{scope{\U}parent}}
\newcommand{\cscoperoot}{\cc{root}}
\newcommand{\chere}{\cc{here}}
\newcommand{\cregion}{\cc{region}}
\newcommand{\cmalloc}{\cc{malloc}}
\newcommand{\cfree}{\cc{free}}

\newcommand{\cbundle}{\cc{bundle}}
\newcommand{\cmessage}{\cc{message}}
\newcommand{\ccomm}{\cc{comm}}
\newcommand{\ccommcreate}{\cc{comm{\U}create}}
\newcommand{\cgcomm}{\cc{gcomm}}
\newcommand{\cgcommcreate}{\cc{gcomm{\U}create}}

\newcommand{\cbarrier}{\cc{barrier}}
\newcommand{\cgbarrier}{\cc{gbarrier}}
\newcommand{\cgbarriercreate}{\cc{gbarrier{\U}create}}
\newcommand{\cbarriercreate}{\cc{barrier{\U}create}}
\newcommand{\cbarrierenter}{\cc{barrier{\U}enter}}
\newcommand{\cbarrierexit}{\cc{barrier{\U}exit}}
\newcommand{\cbarriercall}{\cc{barrier{\U}call}}
\newcommand{\cbarrierdestroy}{\cc{barrier{\U}destroy}}

\newcommand{\cforall}{\cc{forall}}
\newcommand{\cexists}{\cc{exists}}
\newcommand{\cimplies}{\ct{=>}}

\newcommand{\crange}{\cc{range}}
\newcommand{\cdomain}{\cc{domain}}
\newcommand{\cdomainof}[1]{\cc{domain}\ct{(}#1\ct{)}}

% version information
\newcommand{\version}{v0.11}
\newcommand{\versiondate}{2014-04-22}
