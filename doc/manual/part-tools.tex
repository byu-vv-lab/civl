\part{Tools}
\label{part:tools}

\chapter{Introduction}

\section{Symbolic execution}

The tools currently in the CIVL tool kit all use \emph{symbolic
  execution}.  This is a technique in which variables are assigned
symbolic rather than concrete values.  In particular, input variables
are assigned unique \emph{symbolic constants}, which are symbols such
as $X$, $Y$, and so on.  Operations produce symbolic expressions in
those symbols, such as $(X+Y)/2$.

\section{Tool overview}

Current tools allow one to \emph{run} a CIVL program using random
choice to resolve nondeterministic choices; \emph{verify} a program
using model checking to explore all states of the program; and
\emph{replay} a trace if an error is found.  There are also commands
to show the results just of preprocessing or parsing a file; as these
are basically sub-tasks of the other tasks, they are used mainly for
debugging.

Each tool is launched from the command line by typing ``\texttt{civl}
\textit{cmd} \ldots'', where \textit{cmd} is one of
\begin{itemize}
\item \ct{help} : print usage information
\item \ct{run}  : run the program using random simulation
\item \ct{verify} : verify program
\item \ct{replay} : replay trace for program
\item \ct{parse} : show result of preprocessing and parsing file
\item \ct{preprocess} : show result of preprocessing file only.
\end{itemize}
The additional arguments and options are described below and are also
shown by the \ct{help} command.

A number of properties are checked when running or verifying
a CIVL program.  These include the following:
\begin{itemize}
\item absence of deadlocks
\item absence of assertion violations
\item absence of division or modulus with denominator $0$
\item absence of illegal pointer dereferences
\item absence of out-of-bounds array indexes
\item absence of invalid casts
\item every object is defined (i.e., initialized) before
  it is used.
\end{itemize}


\section{ Options}

The following command line options are available:

\newenvironment{optionlist}{\begin{flushleft}}{\end{flushleft}}
\newcommand{\option}[2]{#1\\[0mm]\makebox[5mm]{}#2\\[3mm]}
\newcommand{\booloption}[3]{\option{\ct{-#1} or \ct{-#1=}\textit{boolean}
    (default: \ct{#2})}{#3}}
\newcommand{\intoption}[3]{\option{\ct{-#1=}\textit{integer}
    (default: \ct{#2})}{#3}}
\newcommand{\mapoption}[2]{\option{\ct{-#1}\textit{key}\ct{=}\textit{value}}{#2}}
\newcommand{\stringoption}[3]{\option{\ct{-#1=}\textit{string}
    (default: \ct{#2})}{#3}}

\begin{optionlist}
  \booloption{debug}{false}{debug mode: print very detailed information}
  \booloption{echo}{false}{print the command line}
  \intoption{errorBound}{1}{stop after finding this many errors}
  \booloption{guided}{false}{user guided simulation; applies only to
    run, ignored for all other commands}
  \intoption{id}{0}{ID number of trace to replay}
  \mapoption{input}{initialize input variable \textit{key} to
    \textit{value}}
  \intoption{maxdepth}{2147483647}{bound on search depth}
  \booloption{min}{false}{search for minimal counterexample}
  \stringoption{por}{std}{partial order reduction (por) choices:
    std (standard por) or scp (scoped por)}
  \booloption{random}{varies}{select enabled transitions randomly;
    default for \texttt{run}, ignored for all other commands}
  \booloption{saveStates}{true}{save states during depth-first search}
  \stringoption{seed}{none}{set the random seed; applies only to run}
  \booloption{showModel}{false}{print the model}
  \booloption{showProverQueries}{false}{print theorem prover queries
    only}
  \booloption{showQueries}{false}{print all queries}
  \booloption{showSavedStates}{false}{print saved states only}
  \booloption{showStates}{false}{print all states}
  \booloption{showTransitions}{false}{print transitions}
  \booloption{simplify}{true}{simplify states?}
  \booloption{solve}{false}{try to solve for concrete counterexample}
  \stringoption{sysIncludePath}{}{set the system include path}
  \stringoption{trace}{}{filename of trace to replay}
  \stringoption{userIncludePath}{}{set the user include path}
  \booloption{verbose}{false}{verbose mode}
\end{optionlist}

\chapter{Running a CIVL-C program: \texttt{civl run}}

The syntax for the \ct{run} command is:
\begin{flushleft}
\ \ \ct{civl\ run}\ \textit{options} \textit{filename}
\end{flushleft}

\chapter{Transitions}

In CIVL, a transition stands for the execution of a number of
statements of a certain process from one state, and the execution of
each statement is considered as one step. A transition has the
following form in the output, where sid and sid' are the identifiers
for the source and the target states, pid is the identifier of the
process that this transition belongs to, and step 0,1 ... are the
steps contained in the transitions.

\begin{verbatim}
State sid, proc pid: 
  step 0;
  step 1;
  ...
--> State sid'
\end{verbatim}

A step of a transition has the following form in the CIVL output,
where src and dst is the source and destination location of the
statement executed in the step, file is the file that contains the
source code of the statement, location and text are the location and
summary of text of the statement in the source file.

\begin{verbatim}
  src->dst: statement at file:location "text";
\end{verbatim}

For example, the following is a transition from state 0 to state 1,
containing 6 steps. File names are renamed with short names and the
mapping of short names and the original files is presented in output
as well.

\begin{verbatim}
File name list:
f0	: atomicBlockedResume.cvl

State 0, proc 0: 
  0->1: x = 1 at f0:3.0-9 "int x = 1";
  1->2: y = 0 at f0:4.0-9 "int y = 0";
  2->3: z = 0 at f0:5.0-9 "int z = 0";
  3->6: sum = 0 at f0:6.0-11 "int sum = 0";
  6->8: $spawn foo(1) at f0:22.4-17 "$spawn foo(1)";
  8->10: $spawn foo(2) at f0:23.4-17 "$spawn foo(2)";
--> State 1
\end{verbatim}

To make use of this feature, one needs to specify the option
\texttt{-showTransitions} when running CIVL.

